%!TeX root=../annetop.tex
\chapter{A Queen's Girl}

\lettrine[]{T}{he} next three weeks were busy ones at Green Gables, for Anne was getting ready to go to Queen's, and there was much sewing to be done, and many things to be talked over and arranged. Anne's outfit was ample and pretty, for Matthew saw to that, and Marilla for once made no objections whatever to anything he purchased or suggested. More—one evening she went up to the east gable with her arms full of a delicate pale green material.

<Anne, here's something for a nice light dress for you. I don't suppose you really need it; you've plenty of pretty waists; but I thought maybe you'd like something real dressy to wear if you were asked out anywhere of an evening in town, to a party or anything like that. I hear that Jane and Ruby and Josie have got <evening dresses,> as they call them, and I don't mean you shall be behind them. I got Mrs.~Allan to help me pick it in town last week, and we'll get Emily Gillis to make it for you. Emily has got taste, and her fits aren't to be equalled.>

<Oh, Marilla, it's just lovely,> said Anne. <Thank you so much. I don't believe you ought to be so kind to me—it's making it harder every day for me to go away.>

The green dress was made up with as many tucks and frills and shirrings as Emily's taste permitted. Anne put it on one evening for Matthew's and Marilla's benefit, and recited <The Maiden's Vow> for them in the kitchen. As Marilla watched the bright, animated face and graceful motions her thoughts went back to the evening Anne had arrived at Green Gables, and memory recalled a vivid picture of the odd, frightened child in her preposterous yellowish-brown wincey dress, the heartbreak looking out of her tearful eyes. Something in the memory brought tears to Marilla's own eyes.

<I declare, my recitation has made you cry, Marilla,> said Anne gaily stooping over Marilla's chair to drop a butterfly kiss on that lady's cheek. <Now, I call that a positive triumph.>

<No, I wasn't crying over your piece,> said Marilla, who would have scorned to be betrayed into such weakness by any poetry stuff. <I just couldn't help thinking of the little girl you used to be, Anne. And I was wishing you could have stayed a little girl, even with all your queer ways. You've grown up now and you're going away; and you look so tall and stylish and so—so—different altogether in that dress—as if you didn't belong in Avonlea at all—and I just got lonesome thinking it all over.>

<Marilla!> Anne sat down on Marilla's gingham lap, took Marilla's lined face between her hands, and looked gravely and tenderly into Marilla's eyes. <I'm not a bit changed—not really. I'm only just pruned down and branched out. The real me—back here—is just the same. It won't make a bit of difference where I go or how much I change outwardly; at heart I shall always be your little Anne, who will love you and Matthew and dear Green Gables more and better every day of her life.>

Anne laid her fresh young cheek against Marilla's faded one, and reached out a hand to pat Matthew's shoulder. Marilla would have given much just then to have possessed Anne's power of putting her feelings into words; but nature and habit had willed it otherwise, and she could only put her arms close about her girl and hold her tenderly to her heart, wishing that she need never let her go.

Matthew, with a suspicious moisture in his eyes, got up and went out-of-doors. Under the stars of the blue summer night he walked agitatedly across the yard to the gate under the poplars.

<Well now, I guess she ain't been much spoiled,> he muttered, proudly. <I guess my putting in my oar occasional never did much harm after all. She's smart and pretty, and loving, too, which is better than all the rest. She's been a blessing to us, and there never was a luckier mistake than what Mrs.~Spencer made—if it was luck. I don't believe it was any such thing. It was Providence, because the Almighty saw we needed her, I reckon.>

The day finally came when Anne must go to town. She and Matthew drove in one fine September morning, after a tearful parting with Diana and an untearful practical one—on Marilla's side at least—with Marilla. But when Anne had gone Diana dried her tears and went to a beach picnic at White Sands with some of her Carmody cousins, where she contrived to enjoy herself tolerably well; while Marilla plunged fiercely into unnecessary work and kept at it all day long with the bitterest kind of heartache—the ache that burns and gnaws and cannot wash itself away in ready tears. But that night, when Marilla went to bed, acutely and miserably conscious that the little gable room at the end of the hall was untenanted by any vivid young life and unstirred by any soft breathing, she buried her face in her pillow, and wept for her girl in a passion of sobs that appalled her when she grew calm enough to reflect how very wicked it must be to take on so about a sinful fellow creature.

Anne and the rest of the Avonlea scholars reached town just in time to hurry off to the Academy. That first day passed pleasantly enough in a whirl of excitement, meeting all the new students, learning to know the professors by sight and being assorted and organized into classes. Anne intended taking up the Second Year work being advised to do so by Miss Stacy; Gilbert Blythe elected to do the same. This meant getting a First Class teacher's license in one year instead of two, if they were successful; but it also meant much more and harder work. Jane, Ruby, Josie, Charlie, and Moody Spurgeon, not being troubled with the stirrings of ambition, were content to take up the Second Class work. Anne was conscious of a pang of loneliness when she found herself in a room with fifty other students, not one of whom she knew, except the tall, brown-haired boy across the room; and knowing him in the fashion she did, did not help her much, as she reflected pessimistically. Yet she was undeniably glad that they were in the same class; the old rivalry could still be carried on, and Anne would hardly have known what to do if it had been lacking.

<I wouldn't feel comfortable without it,> she thought. <Gilbert looks awfully determined. I suppose he's making up his mind, here and now, to win the medal. What a splendid chin he has! I never noticed it before. I do wish Jane and Ruby had gone in for First Class, too. I suppose I won't feel so much like a cat in a strange garret when I get acquainted, though. I wonder which of the girls here are going to be my friends. It's really an interesting speculation. Of course I promised Diana that no Queen's girl, no matter how much I liked her, should ever be as dear to me as she is; but I've lots of second-best affections to bestow. I like the look of that girl with the brown eyes and the crimson waist. She looks vivid and red-rosy; there's that pale, fair one gazing out of the window. She has lovely hair, and looks as if she knew a thing or two about dreams. I'd like to know them both—know them well—well enough to walk with my arm about their waists, and call them nicknames. But just now I don't know them and they don't know me, and probably don't want to know me particularly. Oh, it's lonesome!>

It was lonesomer still when Anne found herself alone in her hall bedroom that night at twilight. She was not to board with the other girls, who all had relatives in town to take pity on them. Miss Josephine Barry would have liked to board her, but Beechwood was so far from the Academy that it was out of the question; so Miss Barry hunted up a boarding-house, assuring Matthew and Marilla that it was the very place for Anne.

<The lady who keeps it is a reduced gentlewoman,> explained Miss Barry. <Her husband was a British officer, and she is very careful what sort of boarders she takes. Anne will not meet with any objectionable persons under her roof. The table is good, and the house is near the Academy, in a quiet neighbourhood.>

All this might be quite true, and indeed, proved to be so, but it did not materially help Anne in the first agony of homesickness that seized upon her. She looked dismally about her narrow little room, with its dull-papered, pictureless walls, its small iron bedstead and empty book-case; and a horrible choke came into her throat as she thought of her own white room at Green Gables, where she would have the pleasant consciousness of a great green still outdoors, of sweet peas growing in the garden, and moonlight falling on the orchard, of the brook below the slope and the spruce boughs tossing in the night wind beyond it, of a vast starry sky, and the light from Diana's window shining out through the gap in the trees. Here there was nothing of this; Anne knew that outside of her window was a hard street, with a network of telephone wires shutting out the sky, the tramp of alien feet, and a thousand lights gleaming on stranger faces. She knew that she was going to cry, and fought against it.

<I won't cry. It's silly—and weak—there's the third tear splashing down by my nose. There are more coming! I must think of something funny to stop them. But there's nothing funny except what is connected with Avonlea, and that only makes things worse—four—five—I'm going home next Friday, but that seems a hundred years away. Oh, Matthew is nearly home by now—and Marilla is at the gate, looking down the lane for him—six—seven—eight—oh, there's no use in counting them! They're coming in a flood presently. I can't cheer up—I don't want to cheer up. It's nicer to be miserable!>

The flood of tears would have come, no doubt, had not Josie Pye appeared at that moment. In the joy of seeing a familiar face Anne forgot that there had never been much love lost between her and Josie. As a part of Avonlea life even a Pye was welcome.

<I'm so glad you came up,> Anne said sincerely.

<You've been crying,> remarked Josie, with aggravating pity. <I suppose you're homesick—some people have so little self-control in that respect. I've no intention of being homesick, I can tell you. Town's too jolly after that poky old Avonlea. I wonder how I ever existed there so long. You shouldn't cry, Anne; it isn't becoming, for your nose and eyes get red, and then you seem all red. I'd a perfectly scrumptious time in the Academy today. Our French professor is simply a duck. His moustache would give you kerwollowps of the heart. Have you anything eatable around, Anne? I'm literally starving. Ah, I guessed likely Marilla `d load you up with cake. That's why I called round. Otherwise I'd have gone to the park to hear the band play with Frank Stockley. He boards same place as I do, and he's a sport. He noticed you in class today, and asked me who the red-headed girl was. I told him you were an orphan that the Cuthberts had adopted, and nobody knew very much about what you'd been before that.>

Anne was wondering if, after all, solitude and tears were not more satisfactory than Josie Pye's companionship when Jane and Ruby appeared, each with an inch of Queen's colour ribbon—purple and scarlet—pinned proudly to her coat. As Josie was not <speaking> to Jane just then she had to subside into comparative harmlessness.

<Well,> said Jane with a sigh, <I feel as if I'd lived many moons since the morning. I ought to be home studying my Virgil—that horrid old professor gave us twenty lines to start in on tomorrow. But I simply couldn't settle down to study tonight. Anne, methinks I see the traces of tears. If you've been crying do own up. It will restore my self-respect, for I was shedding tears freely before Ruby came along. I don't mind being a goose so much if somebody else is goosey, too. Cake? You'll give me a teeny piece, won't you? Thank you. It has the real Avonlea flavour.>

Ruby, perceiving the Queen's calendar lying on the table, wanted to know if Anne meant to try for the gold medal.

Anne blushed and admitted she was thinking of it.

<Oh, that reminds me,> said Josie, <Queen's is to get one of the Avery scholarships after all. The word came today. Frank Stockley told me—his uncle is one of the board of governors, you know. It will be announced in the Academy tomorrow.>

An Avery scholarship! Anne felt her heart beat more quickly, and the horizons of her ambition shifted and broadened as if by magic. Before Josie had told the news Anne's highest pinnacle of aspiration had been a teacher's provincial license, First Class, at the end of the year, and perhaps the medal! But now in one moment Anne saw herself winning the Avery scholarship, taking an Arts course at Redmond College, and graduating in a gown and mortar board, before the echo of Josie's words had died away. For the Avery scholarship was in English, and Anne felt that here her foot was on native heath.

A wealthy manufacturer of New Brunswick had died and left part of his fortune to endow a large number of scholarships to be distributed among the various high schools and academies of the Maritime Provinces, according to their respective standings. There had been much doubt whether one would be allotted to Queen's, but the matter was settled at last, and at the end of the year the graduate who made the highest mark in English and English Literature would win the scholarship—two hundred and fifty dollars a year for four years at Redmond College. No wonder that Anne went to bed that night with tingling cheeks!

<I'll win that scholarship if hard work can do it,> she resolved. <Wouldn't Matthew be proud if I got to be a B\@.A\@.? Oh, it's delightful to have ambitions. I'm so glad I have such a lot. And there never seems to be any end to them—that's the best of it. Just as soon as you attain to one ambition you see another one glittering higher up still. It does make life so interesting.>