%!TeX root=../annetop.tex
\chapter{Vanity and Vexation of Spirit}

\lettrine[]{M}{arilla}, walking home one late April evening from an Aid meeting, realized that the winter was over and gone with the thrill of delight that spring never fails to bring to the oldest and saddest as well as to the youngest and merriest. Marilla was not given to subjective analysis of her thoughts and feelings. She probably imagined that she was thinking about the Aids and their missionary box and the new carpet for the vestry room, but under these reflections was a harmonious consciousness of red fields smoking into pale-purply mists in the declining sun, of long, sharp-pointed fir shadows falling over the meadow beyond the brook, of still, crimson-budded maples around a mirrorlike wood pool, of a wakening in the world and a stir of hidden pulses under the gray sod. The spring was abroad in the land and Marilla's sober, middle-aged step was lighter and swifter because of its deep, primal gladness.

Her eyes dwelt affectionately on Green Gables, peering through its network of trees and reflecting the sunlight back from its windows in several little coruscations of glory. Marilla, as she picked her steps along the damp lane, thought that it was really a satisfaction to know that she was going home to a briskly snapping wood fire and a table nicely spread for tea, instead of to the cold comfort of old Aid meeting evenings before Anne had come to Green Gables.

Consequently, when Marilla entered her kitchen and found the fire black out, with no sign of Anne anywhere, she felt justly disappointed and irritated. She had told Anne to be sure and have tea ready at five o'clock, but now she must hurry to take off her second-best dress and prepare the meal herself against Matthew's return from plowing.

<I'll settle Miss Anne when she comes home,> said Marilla grimly, as she shaved up kindlings with a carving knife and with more vim than was strictly necessary. Matthew had come in and was waiting patiently for his tea in his corner. <She's gadding off somewhere with Diana, writing stories or practising dialogues or some such tomfoolery, and never thinking once about the time or her duties. She's just got to be pulled up short and sudden on this sort of thing. I don't care if Mrs.~Allan does say she's the brightest and sweetest child she ever knew. She may be bright and sweet enough, but her head is full of nonsense and there's never any knowing what shape it'll break out in next. Just as soon as she grows out of one freak she takes up with another. But there! Here I am saying the very thing I was so riled with Rachel Lynde for saying at the Aid today. I was real glad when Mrs.~Allan spoke up for Anne, for if she hadn't I know I'd have said something too sharp to Rachel before everybody. Anne's got plenty of faults, goodness knows, and far be it from me to deny it. But I'm bringing her up and not Rachel Lynde, who'd pick faults in the Angel Gabriel himself if he lived in Avonlea. Just the same, Anne has no business to leave the house like this when I told her she was to stay home this afternoon and look after things. I must say, with all her faults, I never found her disobedient or untrustworthy before and I'm real sorry to find her so now.>

<Well now, I dunno,> said Matthew, who, being patient and wise and, above all, hungry, had deemed it best to let Marilla talk her wrath out unhindered, having learned by experience that she got through with whatever work was on hand much quicker if not delayed by untimely argument. <Perhaps you're judging her too hasty, Marilla. Don't call her untrustworthy until you're sure she has disobeyed you. Mebbe it can all be explained—Anne's a great hand at explaining.>

<She's not here when I told her to stay,> retorted Marilla. <I reckon she'll find it hard to explain that to my satisfaction. Of course I knew you'd take her part, Matthew. But I'm bringing her up, not you.>

It was dark when supper was ready, and still no sign of Anne, coming hurriedly over the log bridge or up Lover's Lane, breathless and repentant with a sense of neglected duties. Marilla washed and put away the dishes grimly. Then, wanting a candle to light her way down the cellar, she went up to the east gable for the one that generally stood on Anne's table. Lighting it, she turned around to see Anne herself lying on the bed, face downward among the pillows.

<Mercy on us,> said astonished Marilla, <have you been asleep, Anne?>

<No,> was the muffled reply.

<Are you sick then?> demanded Marilla anxiously, going over to the bed.

Anne cowered deeper into her pillows as if desirous of hiding herself forever from mortal eyes.

<No. But please, Marilla, go away and don't look at me. I'm in the depths of despair and I don't care who gets head in class or writes the best composition or sings in the Sunday-school choir any more. Little things like that are of no importance now because I don't suppose I'll ever be able to go anywhere again. My career is closed. Please, Marilla, go away and don't look at me.>

<Did anyone ever hear the like?> the mystified Marilla wanted to know. <Anne Shirley, whatever is the matter with you? What have you done? Get right up this minute and tell me. This minute, I say. There now, what is it?>

Anne had slid to the floor in despairing obedience.

<Look at my hair, Marilla,> she whispered.

Accordingly, Marilla lifted her candle and looked scrutinizingly at Anne's hair, flowing in heavy masses down her back. It certainly had a very strange appearance.

<Anne Shirley, what have you done to your hair? Why, it's green!>

Green it might be called, if it were any earthly colour—a queer, dull, bronzy green, with streaks here and there of the original red to heighten the ghastly effect. Never in all her life had Marilla seen anything so grotesque as Anne's hair at that moment.

<Yes, it's green,> moaned Anne. <I thought nothing could be as bad as red hair. But now I know it's ten times worse to have green hair. Oh, Marilla, you little know how utterly wretched I am.>

<I little know how you got into this fix, but I mean to find out,> said Marilla. <Come right down to the kitchen—it's too cold up here—and tell me just what you've done. I've been expecting something queer for some time. You haven't got into any scrape for over two months, and I was sure another one was due. Now, then, what did you do to your hair?>

<I dyed it.>

<Dyed it! Dyed your hair! Anne Shirley, didn't you know it was a wicked thing to do?>

<Yes, I knew it was a little wicked,> admitted Anne. <But I thought it was worth while to be a little wicked to get rid of red hair. I counted the cost, Marilla. Besides, I meant to be extra good in other ways to make up for it.>

<Well,> said Marilla sarcastically, <if I'd decided it was worth while to dye my hair I'd have dyed it a decent colour at least. I wouldn't have dyed it green.>

<But I didn't mean to dye it green, Marilla,> protested Anne dejectedly. <If I was wicked I meant to be wicked to some purpose. He said it would turn my hair a beautiful raven black—he positively assured me that it would. How could I doubt his word, Marilla? I know what it feels like to have your word doubted. And Mrs.~Allan says we should never suspect anyone of not telling us the truth unless we have proof that they're not. I have proof now—green hair is proof enough for anybody. But I hadn't then and I believed every word he said implicitly.>

<Who said? Who are you talking about?>

<The peddler that was here this afternoon. I bought the dye from him.>

<Anne Shirley, how often have I told you never to let one of those Italians in the house! I don't believe in encouraging them to come around at all.>

<Oh, I didn't let him in the house. I remembered what you told me, and I went out, carefully shut the door, and looked at his things on the step. Besides, he wasn't an Italian—he was a German Jew. He had a big box full of very interesting things and he told me he was working hard to make enough money to bring his wife and children out from Germany. He spoke so feelingly about them that it touched my heart. I wanted to buy something from him to help him in such a worthy object. Then all at once I saw the bottle of hair dye. The peddler said it was warranted to dye any hair a beautiful raven black and wouldn't wash off. In a trice I saw myself with beautiful raven-black hair and the temptation was irresistible. But the price of the bottle was seventy-five cents and I had only fifty cents left out of my chicken money. I think the peddler had a very kind heart, for he said that, seeing it was me, he'd sell it for fifty cents and that was just giving it away. So I bought it, and as soon as he had gone I came up here and applied it with an old hairbrush as the directions said. I used up the whole bottle, and oh, Marilla, when I saw the dreadful colour it turned my hair I repented of being wicked, I can tell you. And I've been repenting ever since.>

<Well, I hope you'll repent to good purpose,> said Marilla severely, <and that you've got your eyes opened to where your vanity has led you, Anne. Goodness knows what's to be done. I suppose the first thing is to give your hair a good washing and see if that will do any good.>

Accordingly, Anne washed her hair, scrubbing it vigorously with soap and water, but for all the difference it made she might as well have been scouring its original red. The peddler had certainly spoken the truth when he declared that the dye wouldn't wash off, however his veracity might be impeached in other respects.

<Oh, Marilla, what shall I do?> questioned Anne in tears. <I can never live this down. People have pretty well forgotten my other mistakes—the liniment cake and setting Diana drunk and flying into a temper with Mrs.~Lynde. But they'll never forget this. They will think I am not respectable. Oh, Marilla, <what a tangled web we weave when first we practise to deceive.> That is poetry, but it is true. And oh, how Josie Pye will laugh! Marilla, I cannot face Josie Pye. I am the unhappiest girl in Prince Edward Island.>

Anne's unhappiness continued for a week. During that time she went nowhere and shampooed her hair every day. Diana alone of outsiders knew the fatal secret, but she promised solemnly never to tell, and it may be stated here and now that she kept her word. At the end of the week Marilla said decidedly:

<It's no use, Anne. That is fast dye if ever there was any. Your hair must be cut off; there is no other way. You can't go out with it looking like that.>

Anne's lips quivered, but she realized the bitter truth of Marilla's remarks. With a dismal sigh she went for the scissors.

<Please cut it off at once, Marilla, and have it over. Oh, I feel that my heart is broken. This is such an unromantic affliction. The girls in books lose their hair in fevers or sell it to get money for some good deed, and I'm sure I wouldn't mind losing my hair in some such fashion half so much. But there is nothing comforting in having your hair cut off because you've dyed it a dreadful colour, is there? I'm going to weep all the time you're cutting it off, if it won't interfere. It seems such a tragic thing.>

Anne wept then, but later on, when she went upstairs and looked in the glass, she was calm with despair. Marilla had done her work thoroughly and it had been necessary to shingle the hair as closely as possible. The result was not becoming, to state the case as mildly as may be. Anne promptly turned her glass to the wall.

<I'll never, never look at myself again until my hair grows,> she exclaimed passionately.

Then she suddenly righted the glass.

<Yes, I will, too. I'd do penance for being wicked that way. I'll look at myself every time I come to my room and see how ugly I am. And I won't try to imagine it away, either. I never thought I was vain about my hair, of all things, but now I know I was, in spite of its being red, because it was so long and thick and curly. I expect something will happen to my nose next.>

Anne's clipped head made a sensation in school on the following Monday, but to her relief nobody guessed the real reason for it, not even Josie Pye, who, however, did not fail to inform Anne that she looked like a perfect scarecrow.

<I didn't say anything when Josie said that to me,> Anne confided that evening to Marilla, who was lying on the sofa after one of her headaches, <because I thought it was part of my punishment and I ought to bear it patiently. It's hard to be told you look like a scarecrow and I wanted to say something back. But I didn't. I just swept her one scornful look and then I forgave her. It makes you feel very virtuous when you forgive people, doesn't it? I mean to devote all my energies to being good after this and I shall never try to be beautiful again. Of course it's better to be good. I know it is, but it's sometimes so hard to believe a thing even when you know it. I do really want to be good, Marilla, like you and Mrs.~Allan and Miss Stacy, and grow up to be a credit to you. Diana says when my hair begins to grow to tie a black velvet ribbon around my head with a bow at one side. She says she thinks it will be very becoming. I will call it a snood—that sounds so romantic. But am I talking too much, Marilla? Does it hurt your head?>

<My head is better now. It was terrible bad this afternoon, though. These headaches of mine are getting worse and worse. I'll have to see a doctor about them. As for your chatter, I don't know that I mind it—I've got so used to it.>

Which was Marilla's way of saying that she liked to hear it.

